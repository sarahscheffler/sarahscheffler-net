\documentclass{res} 
\setlength{\textheight}{9.5in} % increase text height to fit on 1-page 

\addtolength{\textwidth}{.5in}
\addtolength{\textheight}{.5in}
\usepackage[margin=0.75in]{geometry}

\usepackage{cite}
\bibliographystyle{simple}
\usepackage{xspace}
\usepackage{hyperref}


\begin{document} 


\name{ { \LARGE Curriculum Vitae: Sarah Ann Scheffler} \\[12pt]}     % the \\[12pt] adds a blank
				        % line after name      

\address{\texttt{sarah.ann.scheffler@gmail.com}\\\texttt{https://sarahscheffler.net}\\\texttt{https://github.com/sarahscheffler}}
\address{~~~~~~~~~~18 Vandeventer Ave \#1\\~~~~~~~~~~~~~~Princeton, NJ 08542\\~~~~~~~~~~~~~~~~~~~~\hspace{-0.049cm}(720) 234 - 6853}
                                  
\begin{resume}


    \vspace{0.1in}
\section{POSITIONS HELD}  
\vspace{0.1in}

    \textbf{Postdoctoral Research Associate, Princeton University Center for Information Technology Policy} \\
    September 2021 - Present

    \vspace{0.1in}
\section{EDUCATION}  
\vspace{0.1in}

    \textbf{Ph.D. in Computer Science, Boston University} \\
    Advisor: Prof. Mayank Varia \\
    Thesis: Decrypting Legal Dilemmas \\
    Computer Science Research Excellence Award \\
    September 2016 - June 2021

    \textbf{B.S. joint in Computer Science and Mathematics, Harvey Mudd College}  \\        
    Departmental Honors in Computer Science \\
    Graduated with Distinction \\
    September 2011 - May 2015

\section{REFEREED CONFERENCE PUBLICATIONS}
\vspace{0.25in}

\newcommand{\foregoneVerif}{[1]\xspace}
\newcommand{\copyrightMDL}{[2]\xspace}
\newcommand{\turboikos}{[3]\xspace}
\newcommand{\booligero}{[4]\xspace}
\newcommand{\foregoneconclusion}{[5]\xspace}
\newcommand{\AEC}{[6]\xspace}
\newcommand{\devicefingerprinting}{[7]\xspace}
\newcommand{\pullingblocks}{[8]\xspace}
\newcommand{\fairsibility}{[9]\xspace}
\newcommand{\emailspam}{[10]\xspace}

\begin{itemize}
\item[\foregoneVerif] A.~Cohen, S.~Scheffler, M.~Varia. \\
\emph{Can the government compel decryption?  Don't trust --- verify}. \\
In ACM CS/Law 2022. \\
\url{https://arxiv.org/abs/2208.02905}.

\item[\copyrightMDL] S.~Scheffler, E.~Tromer, M.~Varia. \\
\emph{Formalizing Human Ingenuity: A Quantitative Framework hor Copyright Law's Substantial Similarity}. \\
In ACM CS/Law 2022. \\
\url{https://arxiv.org/pdf/2206.01230}.

\item[\turboikos] Y.~Gvili, J.~Ha, S.~Scheffler, M.~Varia, Z.~Yang, X.~Zhang. \\
\emph{TurboIKOS: Improved Non-interactive Zero Knowledge with Sublinear Memory.} \\
In Applied Cryptography and Network Security 2021. (Acceptance rate: 19.9\%) \\
\url{https://ia.cr/2021/478}.

\item[\booligero] Y.~Gvili, S.~Scheffler, M.~Varia. \\
\emph{BooLigero: Improved Sublinear Zero Knowledge Proofs for Boolean Circuits.} \\
In Financial Crypto 2021.  (Acceptance rate: 24.2\%) \\
\url{https://ia.cr/2021/121}.

\item[\foregoneconclusion] S.~Scheffler M.~Varia. \\
\emph{Protecting Cryptography against Compelled Self-Incrimination.} \\
In USENIX Security 2021. (Acceptance rate: 18.7\%)\\
\url{https://ia.cr/2020/862}.

\item[\AEC] L.~Alcock, S.~Asif, J.~Bosboom, J.~Brunner, C.~Chen, E.~Demaine, R.~Epstein,
A.~Hesterberg, L.~Hirschfeld, W.~Hu, J.~Lynch, S.~Scheffler, L.~Zhang. \\
\emph{Arithmetic Expression Construction.} \\
In International Symposium on Algorithms and Computation 2020. (Acceptance rate: 31.2\%) \\
\url{https://arxiv.org/abs/2011.11767}.

\item[\devicefingerprinting] J.~Milligan, S.~Scheffler, A.~Sellars, T.~Tiwari, A.~Trachtenberg, M.~Varia.  \\
\emph{Case Study: Disclosure of Indirect Device Fingerprinting in Privacy Policies.} \\
In Socio-Technical Aspects of Security 2019. \\
\url{https://arxiv.org/abs/1908.07965}.

\item[\pullingblocks] J.~Ani, S.~Asif, E.~Demaine, Y.~Diomidov, D.~Hendrickson, J.~Lynch, S.~Scheffler, A.~Suhl. \\
\emph{PSPACE-completeness of Pulling Blocks to Reach a Goal.}  \\
In the Japan Conference on Discrete and Computational Geometry, Graphs, and Games 2019.

\item[\fairsibility] R.~Canetti, A.~Cohen, N.~Dikkala, G.~Ramnarayan, S.~Scheffler, A.~Smith. \\
\emph{From Soft Classifiers to Hard Decisions: How fair can we be?.} \\
ACM Fairness, Accountability, and Transparency 2019. (Acceptance rate: 24\%) \\
\url{https://arxiv.org/abs/1810.02003}.

\item[\emailspam] S.~Scheffler, S.~Smith, Y.~Gilad, S.~Goldberg. \\
\emph{The Unintended Consequences of Email Spam Prevention.} \\
In International Conference on Passive and Active Network Measurement 2018. (Acceptance rate: 40\%) \\
\url{https://link.springer.com/chapter/10.1007/978-3-319-76481-8\_12}.
\end{itemize}

\vspace{0.25in}

\section{JOURNAL PUBLICATIONS}
\vspace{0.25in}

\newcommand{\pullingblocksJIP}{[11]\xspace}

\begin{itemize}
\item[\pullingblocksJIP] J.~Ani, S.~Asif, E.~Demaine, Y.~Diomidov, D.~Hendrickson, J.~Lynch, S.~Scheffler, A.~Suhl. \\
\emph{PSPACE-completeness of Pulling Blocks to Reach a Goal.} \\
In the Journal of Information Processing 2020.\\
\url{https://www.jstage.jst.go.jp/article/ipsjjip/28/0/28_929/_pdf}.
\end{itemize}

\vspace{0.25in}

\section{BOOK CHAPTERS}
\vspace{0.25in}

\newcommand{\privateTranslation}{[12]\xspace}

\begin{itemize}
\item[\privateTranslation] T.~Morrison, S.~Scheffler, B.~Pal, A.~Viand. \\
\emph{Private Outsourced Translation for Medical Data.} \\
In Protecting Privacy through Homomorphic Encryption (2021), pp.~107-116. \\
\url{https://link.springer.com/chapter/10.1007/978-3-030-77287-1_7}.
\end{itemize}

\vspace{0.25in}

\section{WORKS IN SUBMISSION}
\vspace{0.25in}

\newcommand{\csamimprovements}{[13]\xspace}
\newcommand{\eeesok}{[14]\xspace}

\begin{itemize}
\item[\csamimprovements] S.~Scheffler, A.~Kulshrestha, J.~Mayer. \\
\emph{Public Verification for Private Hash Matching: Challenges, Policy Responses, and Protocols.} \\
Under submission at IEEE S\&P 2023.

\item[\eeesok] S.~Scheffler, J.~Mayer. \\
\emph{Systemization of Knowledge: Content Moderation for End-to-End Encryption.} \\
Under submission at at PETS 2023.

\end{itemize}

\vspace{0.25in}

\section{MANUSCRIPTS}
\vspace{0.25in}

\newcommand{\autonomousWeapons}{[15]\xspace}
\newcommand{\bog}{[16]\xspace}

\begin{itemize}
\item[\autonomousWeapons] S.~Scheffler, J.~Ostling. \\
\emph{Dismantling False Assumptions about Autonomous Weapon Systems.} \\
Manuscript.  \\
Won 2nd place in the Student Paper Competition at the ACM CSLaw Student Paper Competition in 2019. \\
\url{https://sarahscheffler.net/Autonomous_Weapons_False_Assumptions.pdf}.

\item[\bog] J.~Hennessey, S.~Scheffler, M.~Varia. \\
\emph{On Resilient Password-Based Key Derivation Functions.} \\
Manuscript.  \\
\url{https://sarahscheffler.net/diskcrypt/draft-2018-bog.pdf}.
\end{itemize}

\vspace{0.25in}

\section{HONORS, AWARDS, AND FELLOWSHIPS}
\vspace{0.1in}
    Google Ph.D. Fellowship (supported my Ph.D. research 2019-2021) \\
    Clare Boothe Luce Graduate Fellowship (supported my Ph.D. research 2017-2019) \\
    Boston University Computer Science Research Excellence Award (2021) \\
    RSA Conference Security Scholar (2020) \\
    ACM CSLaw Student Paper Competition: 2nd Place (2019) \\
    Clinic Team Award, HMC Computer Science Department (2015, awarded for an exceptional capstone project) \\
    International Mathematical Competition in Modeling: Meritorious Winner (2014), Honorable Mention (2015) \\       

\section{INVITED TALKS}
\vspace{0.1in}
Stanford Security Lunch (November, 2022) \\
Microsoft Research Cryptography and Privacy Colloquium (November, 2022) \\
ACM CS/Law (November, 2022) \\
Rutgers Seminar for the REU Summer Program at DIMACS for Mathematics and Computer Science (June, 2022) \\
DIMACS Workshop on the Co-Development of Computer Science and Law (May, 2022) \\
CS+Law Research Presentations through Northwestern Univeristy (May, 2022) \\
USENIX Security (Aug. 2021) \\
Privacy Law Scholars Conference (June 2021) \\
Cryptic Commons Workshop (May 2021) \\
Georgetown Data Co-Ops Meeting (Mar. 2021) \\
Duke Privacy and Security Seminar (Mar. 2021) \\
Stanford Security Seminar (Feb. 2021) \\
Berkeley Cryptography Seminar (Jan. 2021) \\
Winter Security Seminar Series at Carnegie Mellon University (Jan. 2021) \\
Real World Crypto (Jan. 2021) \\
MIT Security Seminar (Dec. 2020) \\
Guest lecturer at ETH Zurich course ``Approaches to Authentication and Security: Views from Law,
Economics, and the Scientific Disciplines'' (Nov. 2020) \\
Northeastern Privacy Scholars Workshop (Nov. 2020) \\
DIMACS Workshop on the Co-Development of Computer Science and Law (Nov. 2020) \\
Boston University Security Seminar (Oct. 2020) \\
Cybersecurity Law and Poicy Scholars Conference (planned Apr. 2020, two papers accepted for
discussion, event postponed to Dec. 2020 due to COVID-19) \\
Bridging Privacy seminar at Berkman Klein Center for Internet and Society (Dec. 2019) \\ 
Cornell Crypto Seminar (Nov. 2019) \\
Carnegie Mellon University AI Seminar (Oct. 2019) \\
Boston University CyberAlliance Seminar (Dec. 2018) \\

\section{PROGRAM COMMITTEES}
\vspace{0.1in}

\textbf{Program Committee:} \\
USENIX Security 2023 \\
Workshop on Foundations of Computer Security 2021

\textbf{Shadow Program Committee:} \\
IEEE S\&P 2020

\textbf{External Reviewer:} \\
USENIX Security 2022 \\
IEEE Security \& Privacy 2021 \\
ISCA Symposium on Security and Privacy in Speech Communication 2021 \\
Theory of Cryptography Conference 2020 \\
Eurocrypt 2020 \\
IEEE Security \& Privacy 2020 \\
ACM Conference on Fairness, Accountability, and Transparency 2020 \\
IEEE Computer Security Foundations Symposium 2018 \\
International Conference on Cryptology and Network Security 2017 \\
International Conference on Information Theoretic Security 2016

\vspace{0.25in}


\section{K12 OUTREACH}
\vspace{0.1in}

\textbf{RACECAR Crash Course:} From Oct. 2018 - Jan. 2019, volunteered as a teaching assistant for
this program to prepare high school students for the Beaver Works Summer Institute RACECAR course in the summer. \\
\textbf{Code Creative:} From Jan. 2017 - Jan. 2018, was a mentor for Code Creative, a computer science education
program for Boston-area high school students who do not have access to a computer science course at their schools.
Was responsible for creating slides and labs, lecturing, organizing, and in-class tutoring.
https://www.codecreative-ll.org/ \\
\textbf{Codebreakers:} In summer 2016, as one of a team of three, created and taught a summer cybersecurity class for
high school girls.  Was responsible for creating the curriculum, creating class material and exercises, and leading
classes.  In 2017, 2018, and 2019, was a guest lecturer.  https://www.bu.edu/lernet/cyber/ \\


\section{TEACHING EXPERIENCE}
\vspace{0.1in}
\textbf{Teaching Fellow:} Applied Cryptography, Boston University Computer Science Department (Spring 2018, Spring 2017) \\
\textbf{Head Grader:}  Linear Algebra (2013) and 
Differential Equations (2013), HMC Department of Mathematics \\
\textbf{Tutor/Grader:} Programming Languages (2014) and Principles of Computer Science (2013) and
Intro to Computer Science (2013), HMC Computer Science Department \\
\textbf{Grader:}  Multivariable Calculus (2013) and Calculus (2012) and Probability and Statistics
(2012), HMC Department of Mathematics \\

\section{TRAVEL GRANTS}
\vspace{0.1in}
Real World Crypto (2020), Crypto (2019, 2018), ACM Symposium on Theory of Computing (2019) \\


\section{WORK EXPERIENCE}
\vspace{0in}
    \begin{tabbing}
   \hspace{3in}\=  \hspace{1.63in}\= \kill % set up two tab positions
    {\bf Assistant Staff} \>~~~~~~~~~~~~MIT Lincoln Laboratory \>~~~~~~~~~~~~~~~~~~~~~~~Sep. 2015 - June 2016\\                       
   \end{tabbing}\vspace{-30pt}      % suppress blank line after tabbing
    Worked in the Secure and Resilient Systems and Technology group within the Cybersecurity and Information Sciences division.  Assisted in the implementation and testing of a library that adds confidentiality and integrity guarantees to the Accumulo database, protecting it against a malicious server or sysadmin.

   \begin{tabbing}
   \hspace{3in}\=  \hspace{1.63in}\= \kill % set up two tab positions
    {\bf Implementing Oblivious RAM} \>~~~~~~~~~~~~~MIT Lincoln Laboratory \>~~~~~~~~~~~~~~~~~~~~~~~~~~~~~~~~~~Summer 2015\\                       
   \end{tabbing}\vspace{-30pt}      % suppress blank line after tabbing
    Designed and implemented an Oblivious RAM for the Accumulo database in Java, to hide a querying client's access patterns from a malicious server as part of a larger project within the Secure and Resilient Systems and Technology group.

   \begin{tabbing}
   \hspace{2in}\=  \hspace{1.63in}\= \kill % set up two tab positions
    {\bf Quantifying Latent Fingerprint Quality} \>~~~~~~~~~~~~~~~~~~~~~~~~~~The MITRE Corporation
    and HMC  \>~~~~~~~~~~~~~~~~~~~~~~~~~~~~~~~~~~~~~~~~~~~Fall 2014 - Spring 2015\\                       
   \end{tabbing}\vspace{-30pt}      % suppress blank line after tabbing
    Worked on a team of four students to design, implement, and test a system that uses image processing and machine learning techniques to evaluate the suitability of crime scene fingerprint images for identification by Automated Fingerprint Identification Systems.

	\begin{tabbing}
	\hspace{3in}\=  \hspace{1.63in}\= \kill % set up two tab positions
	{\bf Statistical Testing of Cryptographic Entropy Sources} \>~~~~~~~~~~~~~~~~~~~~~~~~~~~~~~~~~NIST 
 \>~~~~~~~~~~~~~~~~~~~~~~~~~~~~~~~~~Summer 2014\\      
	\end{tabbing} \vspace{-30pt}
	Worked with Dr.~Allen Roginsky in the Computer Security Division of the National Institute of Standards and Technology (NIST) to improve NIST's statistical tests for entropy sources in cryptographic random number generators.  Also made adjustments to the process for generating large primes for cryptography.
\\




\section{ARTICLES AND BLOG POSTS}  
\vspace{0.1in}

\textbf{Cyber Alliance Blog:} \\         
    (Fewer Than) Five Reasons the Tech Sector Isn't All That Different (Feb 2018) \\
    Google Has Its Fingers In Many Pies, But Isn't Monopolizing Any Individual Pie (July 2018) \\
    Computers are Poor Decision-Makers for Ill-Defined Problems (Aug 2018) \\
    The Skeletons in My Closet Are Styrofoam But It's Really None Of Your Business (Sep 2018) 

\textbf{The Conversation:} \\         
    Artificial intelligence must know when to ask for human help (Mar 2019) 

\textbf{Interviews and posts about my work:} \\
    (Written about my work on the Cyber Alliance Blog) Could math change the law? What the 5th Amendment means in cryptography (Mar 2021) \\
    (Interview with me on the Cyber Alliance Blog) Security vs Privacy -- Why should you have to choose on messaging apps? (Mar 2021) \\




\section{COMPUTER SKILLS}  
\vspace{0.1in}

\textbf{Programming:} Rust, Python, C++, C, Haskell, Java, Prolog\\         
\textbf{Software and Frameworks:} NTL, SCALE-MAMBA (SPDZ-2), Mathematica, Sage, R, Matlab, \LaTeX \\



\section{RELEVANT COURSEWORK}   
\vspace{0.1in}

    \textbf{Cryptography:} Multi-Party Computation at Scale, Cryptography,
    Applied Cryptography, Lattice Cryptography \\
    \textbf{Computer science:} Privacy in Machine Learning, Adaptive Data Analysis, 
    Networks, Security, Malware/Vulnerabilities \\   
    \textbf{Law:} Law and Algorithms (joint between BU, Harvard, UC Berkeley, and Georgetown), National Security and Technology \\
    \textbf{Mathematics:} Abstract Algebra, Probability, Number Theory, Linear Algebra,
    Numerical Analysis  \\



\end{resume}
\end{document}

